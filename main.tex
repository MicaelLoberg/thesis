%Notes to myself:
%ABS:
%   http://abs-models.org/documentation/manual/#sec:algebraic-data-types
%   http://delivery.acm.org/10.1145/3130000/3122848/a76-boer.pdf?ip=130.238.252.224&id=3122848&acc=ACTIVE%20SERVICE&key=74F7687761D7AE37%2EDA50417A8F734F4C%2E4D4702B0C3E38B35%2E4D4702B0C3E38B35&CFID=1012165178&CFTOKEN=22086748&__acm__=1512138297_7db1453bc51abbd04fa38d02cff9f460 Page >= 11
%
%
%
%

\documentclass[10pt]{report}

\usepackage[utf8]{inputenc}
\usepackage{float}
%\setlength{\parindent}{0pt} % no auto indent

% CLICKABLE LINKS IN TOC,CITE,REF, ETC.
%\usepackage[hidelinks]{hyperref}

% TABLES
\usepackage{tabularx,booktabs,multirow,bigdelim} % for big parenthesis on side of table
\usepackage[table]{xcolor}

% GRAPHICS
\usepackage{graphicx}
\graphicspath{{images/}}

% BIBLIOGRAPHY
%\usepackage[bibstyle=nature, citestyle=numeric-comp]{biblatex}
%\usepackage{natbib}
%\bibliographystyle{plain} % plain
\bibliography{references}

% APPENDICES
\usepackage[toc,page]{appendix}

% CHAPTER STYLE
\usepackage{titlesec}
\titleformat{\chapter}{\normalfont\huge}{\thechapter.}{20pt}{\huge\bf}

% TIKZ
\usepackage{tikz}
\usepackage{pgf}
\usetikzlibrary{trees}
\usetikzlibrary{shadows,positioning}

% COLORS
\usepackage{color}
\definecolor{eclipseBlue}{RGB}{42,0.0,255}
\definecolor{eclipseGreen}{RGB}{63,127,95}
\definecolor{eclipsePurple}{RGB}{127,0,85}
\definecolor{codeListingBackground}{gray}{0.95}

% CODE LISTINGS
\usepackage{listings}
\usepackage{caption,subcaption}
\setcounter{tocdepth}{5}
\setcounter{secnumdepth}{3}
\setlength{\parindent}{0pt}
\hangafter=0
\usepackage[parfill]{parskip}
%% Define Language
\lstdefinelanguage{Encore}
{
  morekeywords={
  active,
  case,catch,chain,class,data,
  def,do,
  eos,else,end,exception,
  finally,for,fun,
  get,getNext,
  if,import,in,
  let,linear,
  match,module,new,
  print,println,
  qualified,
  return,require,
  shared,stream,
  then,this,throw,trait,try,typedef,
  unless,unsafe,
  val,var,
  when,where,while,with,
  yield
  },
  sensitive=true, % keywords are not case-sensitive
  morecomment=[l]{--}, % l is for line comment
  morestring=[b]" % defines that strings are enclosed in double quotes
}
\lstdefinelanguage{C}
{
  morekeywords={
  int64_t, void, list_t, if, else, NULL, 
  sealed, trait, =>
  },
  sensitive=true, % keywords are not case-sensitive
  morecomment=[l]{--}, % l is for line comment
  morestring=[b]" % defines that strings are enclosed in double quotes
}

\lstdefinelanguage{Scala}
{
  morekeywords={
  match, case, object, class, extends,
  sealed, trait, =>
  },
  sensitive=true, % keywords are not case-sensitive
  morecomment=[l]{--}, % l is for line comment
  morestring=[b]" % defines that strings are enclosed in double quotes
}

%\lstset{numbers=left,xleftmargin=2em,frame=single,framexleftmargin=1.5em}

\lstset{
%  language={Encore},
  basicstyle=\footnotesize\ttfamily, % Global Code Style
  backgroundcolor=,%\color{codeListingBackground},
  captionpos=b, % Position of the Caption (t for top, b for bottom)
  extendedchars=true, % Allows 256 instead of 128 ASCII characters
  tabsize=1, % number of spaces indented when discovering a tab
  columns=fixed, % make all characters equal width
  keepspaces=true, % does not ignore spaces to fit width, convert tabs to spaces
  showstringspaces=false, % lets spaces in strings appear as real spaces
  breaklines=true, % wrap lines if they don't fit
  frame=none, % e.g. none/tbrl/single/lines. tbrl = draw a frame at the top, right, left and bottom of the listing.
  xleftmargin=2ex,
  framexleftmargin=0pt,
  framexrightmargin=0pt,
  framextopmargin=0pt,
  framexbottommargin=0pt,
  frameround=ffff, % make the frame round at all four corners
  framesep=0pt, % quarter circle size of the round corners
  numbers=left, %left, % show line numbers at the left
  numberstyle=\color{eclipseBlue}\tiny\ttfamily, % style of the line numbers
  numbersep=3pt, % distance between numbers and code
  commentstyle=\color{eclipseGreen}, % style of comments
  keywordstyle=\bf, % style of keywords \color{eclipsePurple}
  stringstyle=, % style of strings \color{eclipseBlue}
  showlines=true,
  linewidth=20cm,
}

\lstdefinestyle{encattachment} {
  language=Encore,
  frame=none,
  captionpos=t,
  numbers=none,
  frame=tbrl,
  framexleftmargin=2pt,
  framexrightmargin=2pt,
  framextopmargin=2pt,
  framexbottommargin=2pt,
}

\newcommand{\KIKO}[1]{\textcolor{red}{\textbf{[Kiko: #1]}}}

% CUSTOM COMMANDS
\def\code#1{\texttt{#1}} % Code-esque text
\def\todo#1{\textcolor{blue}{\{TODO: #1\}}}
\def\name#1{\textsc{#1}} % Proper nouns, names, logos, names of programming languages

\title{
    {Algebraic data types in Encore:}\\
    {reconciling objection-oriented and functional programming}\\
    {Uppsala University}
}

\author{Micael Loberg}
\date{\today}

\begin{document}
\pagenumbering{roman}

\maketitle

\newpage\newpage
%\vspace*{\fill}
{\centering \textit{This page intentionally left blank}\par}
\vspace{\fill}


\chapter*{Abstract}
\par{Encore is an object-oriented programming language that has a focus on implementing concurrent and parallel systems and it is using the Actor model. Unlike traditional object-oriented programming languages objects in Encore has their own thread of control. Encore also has passive objects which lack their own thread of control, these are mainly used to create data structures such as lists or trees and are used inside of active objects. This thesis presents Algebraic Data Types (ADTs) as a complement to passive classes, and explores how this construct normally found in functional languages can be a useful addition to a object-oriented language such as Encore. The implementation take advantage of the fact that the Encore compiler has a desugaring phase where the ADT syntax is transformed to passive classes. The paper also discusses the design of the syntax and the behaviour of Algebraic Data Types with concurrent and parallel systems in mind. Optimization has also been made to how pattern matching is done when used together with Algebraic Data Types.}

\newpage\newpage
%\vspace*{\fill}
{\centering \textit{This page intentionally left blank}\par}
\vspace{\fill}


\tableofcontents

\KIKO{Maybe an introduction similar to the style of Gustav's thesis, where
he gives a brief intro and the project goal. Right after that, the background}

%%%%%%%%%%%%%%%%%%%%%%%%%%%%%%%%%%%%%%%%%%%%%
%
\chapter{Background}
%
%%%%%%%%%%%%%%%%%%%%%%%%%%%%%%%%%%%%%%%%%%%%%
\pagenumbering{arabic}
\setcounter{page}{1}

%%%%%%%%%%%%%%%%%%%%%%%%%%%%%%%%%%%%%%%%%%%%%
%
\label{ch:background}
%
%%%%%%%%%%%%%%%%%%%%%%%%%%%%%%%%%%%%%%%%%%%%%

\section{Encore}
\par{Encore is an object-oriented programming language that is being developed at Uppsala University\cite{Encore}. It has a focus on implementing concurrent and parallel systems and it is using the Actor model. Encore has active objects that have their own thread of control and they can communicate with each other through message passing. Encore also has passive objects; these lack their own thread of control and behaves like objects in more traditional object-oriented programming languages such as C++ or Java. Passive objects are mainly used to create data structures such as lists or trees and are used inside of active objects\ldots} %TODO: write moar stuffs!
\section{Algebraic Data Types}
\par{In functional languages we often use Algebraic data types (ADTs) to describe composite data. Data structures can in type theory be described in terms of products, sums and recursive types. This leads to an algebra for describing data structures (hence Algebraic Data Types). Such data types are common in functional languages, such as ML or Haskell.}

\par{Robert Harper describes \textbf{Products} of types in the book \textit{'Practical Foundations for Programming Languages'} as the following:}

\par{\textit{"The binary \textbf{product} of two types consists of ordered pairs of values, one from each type in the order specified. The associated eliminatory forms are projections, which select the first and second component of a pair. The nullary product, or unit, type consists solely of the unique “null tuple” of no values, and has no associated eliminatory form."}}
\par{An example of a product type is a tuple consisting of a \code{bool} and an \code{int}, \code{(bool, int)}. Assuming an 8 bit integer the \code{int} part of the tuple can assume 256 different values, and a boolean can have two different values (\code{true} and \code{false}). So a tuple of this type can assume a total of 512 different values.}
\par{\textbf{Sums} of types can be seen as the choice of two or more variants of a data structure. Some type \code{Foo} could be defined as the choice between a \code{int} or a \code{bool}. In Haskell this could be defined as \code{data Foo = bool | int}. If an integer can have 256 different values and a boolean two; the type \code{Foo} which is defined as the choice between the two can take on a total of 258 different values.}

%TODO: Make listing work
\par{An example of a data structure that is both a sum and product type can be seen in listing~\ref{lst:e4c_syntax}. It is an example of a polymorphic binary tree implemented using ADTs. Here \code{Tree} is the name of the type and \code{Nil} and \code{Branch} are constructors that are used to create new instances of type \code{Tree} and \code{t} is a type variable. The type \code{Tree} is also recursive as it is defined in terms of itself.}


\begin{lstlisting}[language=Haskell,caption={Binary tree definition in Haskell},label={lst:e4c_syntax}]
data Tree t = Nil
            | Branch t (Tree t) (Tree t)
\end{lstlisting}
%\subsection{ADTs In Functional Languages}% (this section has more become something like 'ADTs vs Classes' :/ )
%\par{In the C programming language sum types can be created using unions, product types can be created with structs and types can be recursive by also holding a pointer to its own type. In the previous chapter we saw how ADTs can be used to create both sum and product types, as well as recursive types. Most functional languages does not all of these other constructs, so the primary method of defining composite data is with Algebraic Data Types}

%\KIKO{I found this paper, which seems relevant: https://dl.acm.org/citation.cfm?doid=1094811.1094814, "Generalized algebraic data types and object-oriented programming"}
%\KIKO{I find this introduction a bit abrupt. Maybe ADTs started in the functional setting to solve a problem and you can continue from there.}
%In most functional languages there are no objects, so the primary method of defining composite data is with Algebraic Data Types.

%\KIKO{What do you mean? This first sentence is not clear.}
%Though ADTs is in no way equivalent with classes. Some of the major differences is mutability. Instance variables in a class is generally mutable while values in an ADT is not. Another distinction is that Algebraic Data Types allow for both sums and products while classes only allows for products.
%\KIKO{Maybe I am mistaken but Animal<t> has subclasses Cat<t> and Dog<t>. Can a polymorphic class be considered sum type? I am not saying that it is, I am asking.}

%ADTs also expose their insides to the world in a way that classes do not. This allows the programmer to perform pattern matching on the structure of an Algebraic Data Type. Pattern matching on the structure of types is also a feature that is mostly found in Functional languages and is critical to work with ADTs an effective manner.
\subsection{ADTs In Imperative/OO Languages}
\subsubsection{Scala}


\par{While Algebraic Data Types is mostly found in functional languages there are examples of imperative languages that features them. Perhaps most notably in Scala.}
\par{A binary tree in Scala could have the following definition:}

\begin{lstlisting}[language=Scala,caption={ADT definition in Scala},label={lst:e4c_syntax}]
sealed trait Tree[+T]
case object Nil extends Tree[Nothing]
case class Branch[T](value:T, left:Tree[T], right:Tree[T]) extends Tree[T]
\end{lstlisting}
\par{Noteworthy here is that the ADT is defined in terms of a Trait and classes extending the trait. In Scala a sealed trait is a special kind of trait that can only be extended in the same file as it is defined. A case class is mostly a normal class with a few differences such as its being immutable\cite{ScalaCase}. Case classes can also be used in pattern matching:}
\begin{lstlisting}[language=Scala,caption={Pattern matching on an ADT in Scala},label=scala-match]
tree match {
    case Branch(value, left, right) => Foo()
    case Nil()  => Bar()
}
\end{lstlisting}
\par{One can easily see the similarities of this implementation and the one in listing \ref{scala-match}. Here \code{Tree} is the new type, \code{Nil} and \code{Branch} could be seen as constructors for the type \code{Tree}.}
\par{By using the scheme of traits and classes to create something that looks and behaves a lot like an ADT Scala have created objects and are both sum and product types, and the fact that case classes can be used in destructured pattern matching allows for allows for programming patterns normally found only in functional languages.}
\subsubsection{ABS}
\par{The Abstract Behavioral Specification language\cite{ABS} (ABS) is another language that features ADTs. It is a language that is interesting to look at because just like Encore it is an Object Oriented language that uses the Actor model as its concurrency model. When implementing ADTs in Encore, some design decisions have to be made and looking at another language that were faced with the same design decisions might give some insight of what a good design could be. One such example is that of mutability. In ABS all values held by an ADT is immutable, the reasoning behind this is to make them safe to pass around between different actors and that it makes it easier to reason about the code\cite{ABSmut}}
\par{A linked list containing Integers can be defined as}
\begin{lstlisting}[language=encore,caption={Linked list in ABS}]
data List = Nil
          | Node(Int value, List tail);
\end{lstlisting}
\par{Names for constructors and their arguments can optionally be omitted, an example is the following implementation of the Bool datatyp\}

% KIKO: I would recomment that listings have a \label as well, e.g.
\begin{lstlisting}[language=encore,caption={Actual definition of built-in type Bool},label=test-kiko]
data Bool = True | False;
\end{lstlisting}

\begin{lstlisting}[language=encore,caption={Actual definition of built-in type Bool}]
data Bool = True | False;
\end{lstlisting}
\par{If data constructor arguments have names, like \code{head} and \code{tail} in listing \ref{abs-list} - it defines a function that, when passed a value expressed with the given constructor, return the argument.  The name of an accessor function must be unique in the module it is defined in. It is an error to have multiple accessor functions with the same name, or to have a function definition with the same name as an accessor function.}

\begin{lstlisting}[language=encore,caption={Accessor funtion in ABS},label=abs-list]
data List = Nil
          | Node(Int head, List tail);
{
  List list = List(1, List(2, Nil));
  Int head = head(list); #Variable head is assigned the value 1
}
\end{lstlisting}


\KIKO{Grammar-wise, be careful when you talk in singular and plural: e.g. ADTs is in no way equivalent with classes -- it should be -- ADTs are in no way equivalent with classes. I have seen this mistake more than
5 times in chapter 1, and it happens from the singular to plural case and vice versa. Use "is" when you talk about a single thing and "are" when you talk about many things :)}

\chapter{Encoding ADTs in Encore}
\subsection{Encoding scheme}
\par{In Encore it is currently possible to emulate the behaviour of ADTs using a scheme of traits and classes.}\cite{gustavL}%TODO: add reference to Gustav Lundins paper
\par{An implementation of a linked list in Encore could consist of the two classes \code{Node} and \code{Last}, both of which implements a trait \code{LinkedList}}
\par{Classes can implement extractor methods that exposes the internal structure of an object which can then be used in pattern matching. If the trait \code{LinkedList} requires the classes who implements the trait to also implement destructor methods, an object of type \code{LinkedList} can be used in pattern matching}

\begin{lstlisting}[language=encore,caption={trait LinkedList that requires classes to implement destructor methods}]
trait LinkedList[t]
  require def Node() : Maybe[(t, LinkedList)]
  require def Last() : Maybe[(t)]
end
\end{lstlisting}
\par{The class \code{Node} in listing \ref{encoding-scheme} will implement both of the extractor methods \code{Node()} and \code{Last()}. The method \code{Node()} will return an option type containing a tuple consisting of the internal objects to be exposed, in this case \code{value} and \code{next} while the method \code{Last()} will return \code{Nothing}}

\par{Similarly an object of type \code{Last} (listing \ref{encoding-scheme}) will also implement both extractor methods \code{Node()} and \code{Last()}. Though here the method \code{Node()} will return \code{Nothing} while \code{Last()} returns \code{Just(value).}}

\begin{lstlisting}[language=encore,caption={Implementation of Node and Last classes},label=encoding-scheme]
read class Node[t] : List[t](value, next)
  val value : t
  val next : List[t]

  def init(value : t, next : List[t]) : unit
    this.value = value
    this.next = next
  end

  def Last() : Maybe[t]
    Nothing
  end

  def Node() : Maybe[(t, List[t])]
    Just((this.value, this.next))
  end
end

read class Last[t] : List[t](value)
  val value : t

  def init(value : t) : unit
    this.value = value
  end

  def Last() : Maybe[t]
    Just(this.value)
  end

  def Node() : Maybe[(t, List[t])]
    Nothing
  end
end
\end{lstlisting}

\par{To make the classes \code{Node} and \code{Last} look more like branches of an ADT and hide the fact that they are classes, we also add constructor functions to create the objects for us.}

\begin{lstlisting}[language=encore,caption={Constructor functions for Node and Last}]
fun Node[t](value : t, next : List[t]) : List[t]
  new Node[t](value, next)
end

fun Last[t](value : t) : List[t]
  new Last[t](value)
end
\end{lstlisting}

\par{A LinkedList containing the values $[1, 2, 3]$ can now be created like this}
\begin{lstlisting}[language=encore,caption={Creation of list containing three elements}]
var list = Node(1, Node(2, Last(3)))
\end{lstlisting}

\par{The variable \code{list} can now be used in pattern matching}

\begin{lstlisting}[language=encore,caption={Function that uses pattern matching to calculate the length of a list}]
fun listLength[t](list : List[t]) : int
  match list with
    case Node(value, next) => 1 + listLength(next)
    case Last(value) => 1
  end
end
\end{lstlisting}

\par{Now we have something that looks and behaves a lot like an ADT in for example Haskell. We have functions that creates the objects for us and hides the fact that we're creating an object from a class and we can perform pattern matching on the structure of of the object.}

\subsection{Problems with encoding} \label{problems}
\par{In the previous section we have seen how we in Encore can emulate the behavior of ADTs by using a scheme of traits, classes, methods and pattern matching. Though this is far from perfect.}
\par{Perhaps the most noticeable problem is the difference in amount of lines of code needed to create something that behaves like an ADT in Encore compared to languages that has ADTs as actual types.}
\par{Another problem is the cost of doing pattern matching on classes. Consider the extractor method \code{Node()} on line 30 in listing~\ref{encoding-scheme}. The purpose of the method is to tell if a given \code{List} is of type \code{Node} or not. The value it returns is a \code{Maybe} type, holding a tuple that contains the values a \code{Node} wants to expose. Memory for both the Maybe and the Tuple types needs to be allocated for and the fields from the Node needs to be copied over to the tuple. All of this adds a considerable amount to the runtime of an application that does a lot of pattern matching.}
\par{Given these problems we fix them by extending Encore and adding ADTs to the language. A more compact syntax and a better optimized method of performing pattern matching is designed.}
\chapter{Design}
TODO

\KIKO{Maybe in this phase you can explain the phases that the compiler has and
  why you decided to do things in the desugaring and code gen phases. The advantages
of this design choice, etc.}

\subsection{Syntax}
\par{The syntax for ADTs have gone through a number of iterations before we settled on the final one described below.  It is inspired by the syntax Scala use for its ADTs, but modified to fit in with the rest of the Encore language. The grammar of the syntax can be seen in listing~\ref{BNF} in Extended BNF form\cite{eBNF}}%TODO: wikipedia ref to ebnf

\begin{lstlisting}[language=Encore,caption={Grammar for the suggested syntax},label=BNF,mathescape=true]
AdtDeclaration ::= "data", AdtIdentifier, Body
AdtIdentifier ::= Identifier, TypeParams
Identifier ::= (*any capitalized word*)

TypeParams ::= $\epsilon$ | "[", Param, "]"
Param ::= varName | typeVar, ",", Param
varName ::= (*any lower case word*)

ADTBranch ::= "case", AdtIdentifier, "("Fields")",
                ":", AdtIdentifier, Body
Body ::= $\epsilon$ | MethodDeclaration, {MethodDeclaration}, "end"
Fields ::= $\epsilon$ | Field
Field  ::= varName, ":" Type, | varName, ":", Type, ",", Field
\end{lstlisting}

\par{An ADT is defined by using the keyword \code{data}, followed by an identifier starting with a capital letter.}

\begin{lstlisting}[language=Encore]
data Foo
\end{lstlisting}
A branch (or constructor?) of an ADT is defined as following:
\begin{lstlisting}[language=Encore]
case Bar(valueA : t1, valueB : t2) : Foo
\end{lstlisting}
\par{where \code{Bar} is the name of the branch, \code{valueA} and \code{valueB} are the fields of type t1 and t2, \code{:Foo} lets the compiler know that \code{Bar} is a branch of \code{Foo}.
A branch does not have to be defined on the line following the ADT, but anywhere as long as it is in the same file.
Both ADTs and its branches can define methods. Methods are defined on a new indented line under the ADT/branch definition.
An ADT or branch that have methods defined needs to be closed with the \code{end} keyword.}
\begin{lstlisting}[language=encore,caption={ADT definition with a method}]
data Foo
  def Bar() : unit
    --methodbody
  end
end
\end{lstlisting}
\par{ADTs can also take optional type parameters. Type parameters are defined with a comma separated list within brackets}
\begin{lstlisting}[language=encore,caption={Generic linked list implemented with an ADT}]
data List[t]
case Node[t](value : t, next : List[t]) : List[t]
case Nil[t]() : List[t]
\end{lstlisting}
\par{To create an instance of a branch you call the constructor functions that is generated for each of the branches.
A instance of a \code{Node} can be created like this:}
\begin{lstlisting}[language=encore,caption={Declaration of a list containing one element}]
let
  list = Node(1, Nil())
in
  --body
end
\end{lstlisting}
\par{ADTs can be used in pattern matching expressions}
\begin{lstlisting}[language=encore,caption={Pattern matching on a linked list}]
match list with
  case Node(value, next) => Foo()
  case Nil() => Bar()
end
\end{lstlisting}
\subsection{Behaviour}
\par{ADTs and their branches gets desugared to read only traits and classes, so they can do everything traits and classes are capable of.  Methods that are declared inside of the ADT declaration ends up in the trait as required methods, and methods declared in the branch end up in the class. It's however worth to note that in the current state of Encore, when you call a constructor function for a branch you will get an object with the type of the trait back. So right now it's not possible to ever call a method on an ADT branch. This can quite easily be solved by adding a few more features to Encore, this I will discuss in chapter 6.}
\par{As mentioned above, the classes and traits generated are read only, this means that the values held by an ADT branch are immutable.  The main motivation for them being immutable is that it is what I believe most users will expect from an ADT as its a language feature mostly found in functional languages. It also makes them safe to pass around between different actors as no actor is able to modify it.}
\chapter{Implementation}

\subsection{Implementation via desugaring}
\par{The Encore compiler is written in Haskell and generates C code as output which is then piped into Clang to generate executable code. The Encore compiler has a desugaring phase that can be used to turn the ADT nodes in the Abstract Syntax Tree (AST) into class and trait nodes. Methods that are used as constructors for the different branches will also be created.}
\par{The following linked list implemented with an ADT}

\begin{lstlisting}[language=encore,caption={Linked list before it has been desugared}]
data List[t]
case Node[t](value : t, next : List[t]) : List[t]
case Last[t](value : t) : List[t]
\end{lstlisting}

\par{will after the desugaring phase be transformed to the following trait, classes and methods.}

\begin{lstlisting}[language=encore,caption={Desugared linked list},label=desugared]
read trait List[t]
  require def Last() : Maybe[t]
  require def Node() : Maybe[(t, List[t])]
end

read class Node[t] : List[t](value, next)
  val value : t
  val next : List[t]

  def init(value : t, next : List[t]) : unit
    this.value = value
    this.next = next
  end

  def Last() : Maybe[t]
    Nothing
  end

  def Node() : Maybe[(t, List[t])]
    Just((this.value, this.next))
  end
end

read class Last[t] : List[t](value)
  val value : t

  def init(value : t) : unit
    this.value = value
  end

  def Last() : Maybe[t]
    Just(this.value)
  end

  def Node() : Maybe[(t, List[t])]
    Nothing
  end
end

fun Node[t](value : t, next : List[t]) : List[t]
  new Node[t](value, next)
end

fun Last[t](value : t) : List[t]
  new Last[t](value)
end

\end{lstlisting}

\par{The trait \code{List} that has been generated in listing~\ref{desugared} requires that the classes implements extractor methods, one for each branch of the ADT\@. The extractor methods are used in pattern matching and their purpose will be discussed in the next chapter. In this case the extractor methods are \code{Node()} and \code{Last()}.}

\par{Every branch in the ADT will be transformed into a class containing the fields contained in the branch, a constructor method and extractor methods for all the branches of the ADT.}

\par{A creator function for each branch will also be created. These are used as a syntactic sugar to create instances of the ADTs branches and hide the fact that they are implemented as classes.}

\subsection{Pattern matching optimization}
\par{In section~\ref{problems} we see why the current implementation of pattern matching in Encore is not optimal for use with ADTs. Here we will look at the improvements that has been made in this area.}
\par{Consider the following ADT definition}

\begin{lstlisting}[language=encore]
data List
case Node(value : int, next : List) : List
case Last(value : int) : List
\end{lstlisting}

\par{Prior to any of my optimizations the extractor methods for \code{Node} would look like the following:}

\begin{lstlisting}[language=encore,caption={Extractor methods before optimization}]
def Node() : Maybe[(int, List)]
  Just((this.value, this.next))
end

def Last() : Maybe[int]
  Nothing
end
\end{lstlisting}

\par{As explained in section~\ref{problems} the problem with this is the return types of the extractor methods. The value it returns is a \code{Maybe} type, holding a tuple that contains the values a \code{Node} wants to expose. Memory for both the Maybe and the Tuple types needs to be allocated for and the fields from the Node needs to be copied over to the tuple.}


\begin{lstlisting}[language=encore,caption={Pattern matching on a List},label=listmatch]
match list with
  case Node(value, next) => foo()
  case Last(value) => bar()
end
\end{lstlisting}

\par{The code generated from the pattern matching in listing~\ref{listmatch} will generate C code that resembles that of listing~\ref{matchgen}.}


\begin{lstlisting}[language=encore,caption={Pattern matching on a List},label=matchgen]
  insert horrible code here
\end{lstlisting}

\par{A more efficient way of doing this is to get rid of the option type associated with the extractor methods and instead have them return an integer, 1 if it is a match, 0 otherwise. To avoid having to allocate a tuple and copy values to and from it we instead we make us of typecasts so that the fields can be accessed directly. The code from listing~\ref{listmatch} now generates the following C code}

\begin{lstlisting}[language=encore,caption={Pattern matching on a List},label=matchgen2]
  insert slightly less horrible code here
\end{lstlisting}
\par{By getting rid of the Maybe and Tuple types we make avoid allocating memory on the heap that the garbage collector would have to clean up afterwards.}
\chapter{Evaluation and discussion}
TODO
\section{Expressive power}
TODO
\subsection{Some cool example}
TODO
\subsection{Performance}
TODO
\subsubsection{Benchmark}
TODO
\chapter{Conclusion and Future work}
\par{ADTs in Encore are cool and/or totally sweet!}

%#Källor:
%http://www.cs.cmu.edu/~rwh/pfpl.html
%\printbibliography

\chapter{References}


\begin{thebibliography}{9}
\bibitem{gustavL}
Gustav Lundin
\textit{Pattern Matching in Encore}.
http://www.diva-portal.org/smash/get/diva2:930151/FULLTEXT01.pdf

\bibitem{eBNF}
Wikipedia
\textit{Extended BNF}
https://en.wikipedia.org/wiki/Extended\_Backus\%E2\%80\%93Naur\_form

\bibitem{ABS}
Einar Broch JohnsenReiner HähnleJan SchäferRudolf SchlatteMartin Steffen
\textit{ABS: A Core Language for Abstract Behavioral Specification}
https://link.springer.com/chapter/10.1007\%2F978-3-642-25271-6\_8

\bibitem{ABSmut}
\textit{Algebraic Data Types in ABS}
http://abs-models.org/documentation/manual/\#sec:algebraic-data-types

\bibitem{ScalaCase}
\textit{Case classes in Scala}
https://docs.scala-lang.org/tour/case-classes.html

\bibitem{Encore}
Stephan Brandauer Elias Castegren Dave Clarke Kiko Fernandez-Reyes Einar Broch JohnsenKa I. PunS. Lizeth Tapia Tarifa Tobias Wrigstad Albert Mingkun Yang
\textit{Parallel Objects for Multicores: A Glimpse at the Parallel Language Encore}

%\bibitem{}
%http://haskell.cs.yale.edu/wp-content/uploads/2011/02/history.pdf
\end{thebibliography}


\begin{appendices}
\end{appendices}
\end{document}
